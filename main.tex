\documentclass[10pt,a4paper]{article}
\usepackage[a4paper,margin=0.75in]{geometry}
\usepackage{resume}
\usepackage[english]{babel}
\hyphenation{super-vision}


\begin{document}
\sloppy

\maintitle{Soumadeep Saha}{\\Senior Research Fellow, \textbf{Indian Statistical Institute, Kolkata}}{Last update on \today}

\nobreakvspace{0.3em}

\noindent\href{mailto:soumadeep.saha\_r@isical.ac.in}{soumadeep.saha\_r@isical.ac.in}\sbull
\noindent\href{mailto:soumadeep.saha97@gmail.com}{soumadeep.saha97@gmail.com}\sbull
\textsmaller{+}91 8697373806\sbull
\href{https://github.com/espressoVi}{github.com/espressoVi}\\
New Town, Kolkata - 700161, India

\spacedhrule{0.9em}{-0.4em}

\roottitle{Summary}  % a root section title

\vspace{-1.3em}  % some vertical spacing
\begin{multicols}{2}  % open a multicolumn environment
\noindent \emph{My current research is focused on trying to inculcate key strengths of \textbf{symbolic AI} techniques, like \textbf{domain knowledge adherence, logical coherence}, etc into \textbf{deep learning} systems.
	Addition of logical constraints and pre-existing knowledge not only makes these systems more aligned to critical applications but also makes them more data efficient.
	This problem shows up in many domains and thus leads me to work in several fields like natural language, medicine, biology, astrophysics and diverse business applications.}
\\
\\
	I am currently a $3^{rd}$ year PhD candidate at the \href{https://cvpru.isical.ac.in/}{Computer Vision and Pattern Recognition Unit}, \href{https://www.isical.ac.in/}{\textbf{Indian Statistical Institute, Kolkata}} working under the supervision of \textbf{\href{https://www.isical.ac.in/~utpal}{Prof.\;Utpal Garain}}.
\end{multicols}

\spacedhrule{0.3em}{-0.4em}
%--------------------------------- EDUCATION --------------------------------
\roottitle{Education}
\headedsection
  {\href{http://www.iiserkol.ac.in}{Indian Institute of Science Education and Research, Kolkata}}
  {\textsc{Kolkata, India}} {
  \headedsubsection
    {Integrated BS-MS \textnormal{~(Major in Physics, Minor in Mathematics)}}
    {2015 -- 2020}
    {\bodytext{I graduated with a major in \textbf{Physics} and a minor in \textbf{Mathematics}.
    The plethora of advanced Mathematics and Physics courses equipped me with the tools required to tackle today's challenges in the field of Deep Learning and gives me a deeper insight into its inner machinations.
    My master's dissertation dealt with the issue of \textbf{Adversarial Robustness in Deep Learning systems}.
    We found that there is a natural correspondence between the `over-fitting' problem and the lack of robustness.
    We demonstrated that some of the techniques we use to avoid over-fitting also yield better adversarial robustness and that model architecture should be informed by these considerations.\\
    GPA 7.8/10}}
}
\vspace{1em}
\headedsection
  {\href{https://bhavansgkvidyamandir.edu.in/}{Bhavan's G.K. Vidyamandir}}
  {\textsc{Kolkata, India}} {%
  \headedsubsection
    {10+2 \textnormal{~(Pre-University Secondary Education)}}
    {2002 -- 2015}
    {\bodytext{\begin{itemize}
        \item[\ding{111}] Scored $91.2\%$ in CBSE 10th Standard and $92\%$ in Senior Secondary (12th Standard) exams.
		\item[\ding{111}] Recipient of the prestigious National Talent Search Examination (\textbf{NTSE}) Scholarship from NCERT, Govt.\;of India.
		\item[\ding{111}] Recipient of the Kishore Vigyan Pratoshan Yojna (\textbf{KVPY}) fellowship from the Department of Science and Technology, Govt.\;of India.
		%\item[\ding{111}] Trained for olympiads and was selected for the Indian National Olympiad in Informatics.
    \end{itemize}}}
}
\vspace{1em}
\spacedhrule{0.5em}{-0.4em}
%--------------------------------- /EDUCATION -------------------------------
%--------------------------------- PUBLICATIONS -----------------------------
\roottitle{Publications}

\headedsection
  {\ding{111} \href{https://doi.org/10.1371/journal.pone.0283895}{MedTric : A clinically applicable metric for evaluation of multi-label computational diagnostic systems}}
  {\textbf{S. Saha}, U. Garain, A. Ukil, A. Pal, S. Khandelwal,} {%
  \headedsubsection
    {PLOS One, \normalfont{10.1371/journal.pone.0283895}}
    {Accepted March 20, 2023}{}
}
\vspace{0.6em}
\headedsection
  {\ding{111} \href{https://arxiv.org/abs/2212.10696}{Analyzing Semantic Faithfulness of Language Models via Input Intervention on Conversational Question Answering}}
  {A. Chaturvedi, S. Bhar, \textbf{S. Saha}, U. Garain, N. Asher} {%
  \headedsubsection
    {Computational Linguistics, \normalfont{In Press}}
    {Accepted July 17, 2023}{}
}
\vspace{0.6em}
\headedsection
  {\ding{111} \href{https://arxiv.org/abs/2308.05101}{DOST -- Domain Obedient Self-supervised Training for Multi Label Classification with Noisy Labels}}
  {\textbf{S. Saha}, U. Garain, A. Ukil, A. Pal, S. Khandelwal} {%
  \headedsubsection
    {In Press.\normalfont{10.48550/arXiv.2308.05101}}
    {Accepted AAAI 2024 (W3PHIAI);  December 15, 2023}{}
}
\vspace{0.6em}
\headedsection
  {\ding{111} \href{https://arxiv.org/abs/2401.17029}{LADDER: Revisiting the Cosmic Distance Ladder with Deep Learning Approaches and Exploring its Applications}}
  {R. Shah, \textbf{S. Saha}, P. Mukherjee, U. Garain, S. Pal} {%
  \headedsubsection
    {arXiv preprint, \normalfont{10.48550/arXiv.2401.17029}}
    {Submitted to ApJS;  March, 2024}{}
}
\vspace{1em}
\spacedhrule{-0.2em}{-0.4em}
%--------------------------------- /PUBLICATIONS ----------------------------
%--------------------------------- PATENTS ----------------------------------
\roottitle{Patents}
\headedsection{\ding{111} \quad Method and System for Evaluating Clinical Efficiency of Multi-label Multi-class Computational Diagnostic Models}
  {U. Garain, \textbf{S. Saha}, A. Ukil, T. Deb, S. Richa A. Pal, S. Khandelwal}{
  \headedsubsection{Application No. 20221052587}{Filed on 14th September 2022}{}}
\vspace{2em}
\headedsection{\ding{111} \: Method and System for Contradiction Avoided Learning for Multi-class Multi-label \mbox{Classification}}
  {\textbf{S. Saha}, U. Garain, A. Ukil, A.Pal}{
  \headedsubsection{Application No. 202221062230}{Filed on 1st November 2022}{}}
\vspace{1em}
\spacedhrule{-0.2em}{-0.4em}
%--------------------------------- /PATENTS ---------------------------------
%--------------------------------- EXPERIENCE -------------------------------
\roottitle{Experience}

\headedsection{\textbf{Helmholtz Visiting Researcher}}{Jul \apo24 -- Sep \apo24}{
    \bodytext{Recipient of the Helmholtz Information and Data-science Academy (HIDA) visiting researcher grant to work at the Institute of Aerospace Medicine, DLR (German Aerospace).}}

\headedsection{\textbf{TCS Research}}{Nov \apo21 -- Jul \apo22}{
    \bodytext{We worked on diagnosing cardiovascular diseases from ECG signals. I started from scratch, analysing the problem, and pointed out several key challenges that are not yet addressed in the literature, and came up with state of the art solutions, leading to two patents and publications.}}

\headedsection{\textbf{Deep Analysis of Pain Management}}{Jun \apo20 -- Nov \apo20}{
    \bodytext{Collaborated with medical professionals in the field of radiodiagnosis to formulate a problem statement and set up data gathering protocols to create a high quality data set for analysis of back pain from MRI images.}}

\headedsection{\textbf{Lattice Gauge Theory Simulations}}{May \apo19 -- Jul \apo19}{
    \bodytext{I worked on parallelising simulation programs for lattice gauge theory problems using OpenMP, meant to run on cutting edge massively parallel super computers under the supervision of Dr.\.Pushan Majumdar at IACS, Kolkata.}}

\headedsection{\textbf{Teaching and Presentations}}{}{\bodytext{\begin{itemize}
	\item[\ding{111}] Instructor at the Winter School of Deep Learning (WSDL), ISI Kolkata (2021, 2022).
	\item[\ding{111}] Insturctor for the Comprehensive Course on Business Analytics, ISI Kolkata (2022).
	\item[\ding{111}] TA for Natural Language Processing course at ISI Kolkata.
	\item[\ding{111}] Presented my work on Logically Coherent Deep Learning at Amazon Research Days (ARD \apo22)
\end{itemize}}}

\spacedhrule{1em}{-0.4em}

%--------------------------------- /EXPERIENCE ------------------------------
%--------------------------------- SKILLS -----------------------------------
\roottitle{Skills}
	 \ding{111} I am intimately familiar with the state of the art vision (\textbf{ViT, ResNets, etc}) and language models (\textbf{LLMs, BERT, XLNet, etc}).\\
	 \ding{111} In addition to being well-versed in supervised, semi-supervised and unsupervised training paradigms, I am also familiar with techniques like \textbf{adversarial training, fine-tuning, transfer learning, prompting, distillation, etc}.\\
	 \ding{111} Adept at Deep Reinforcement Learning (\textbf{DQN, PPO, MCTS, etc}). Well-versed in time series modelling. \\
	 \ding{111} Significant expertise in Numpy, pyTorch and Python in general.\\
	 \ding{111} Have worked with several programming languages (C++, FORTRAN, bash, etc) and comfortable with Photoshop, GIMP, Premiere, 3D Modelling (blender), CAD (onshape), \LaTeX, HTML/CSS, etc.

\spacedhrule{1.6em}{-0.4em}
%--------------------------------- /SKILLS ----------------------------------
%--------------------------------- INTERESTS --------------------------------
\roottitle{Interests}
	 \ding{111} \textbf{Robotics} - I have an active interest in robotics, be it writing image processing or SLAM algorithms or designing a robot that can climb stairs. I have also conducted introductory workshops on robotics and was the Secretary of the Robotics and Astronomy club at IISER Kolkata.\\
	 \ding{111} \textbf{Music} - Classically trained pianist and enjoy listening to and performing works by Chopin, Beethoven, etc.\\
	 \ding{111} \textbf{Sports} - Represented my college in national level sports meets in Basketball and Volleyball. Played in my state's Senior Division Men's Basketball League.\\
	 \ding{111} Also interested in DIY-ing, gardening, 3D printing, electronics, etc.\\

\spacedhrule{1.6em}{-0.4em}
\end{document}

